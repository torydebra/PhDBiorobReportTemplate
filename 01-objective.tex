\section{Objectives}
%  [max 1 page]
%%Motivate your research and define the general goal of the research project (i.e., which is the addressed S/T question?
%Detail the specific objectives of the current year, and summarize the planned activities of the other two/one years. 

Nowadays, the recent advancements in the development of robotic systems have augmented the robot capabilities, making it possible to face complex tasks. Indeed, in always more challenging scenarios, like industry and construction sites, robots can help the workers, relieving them from the most exhausting and dangerous part of their job, for example the manipulation of heavy loads in unstructured environments. 
The growing complexity of robots and the effective exploitation of their enhanced capabilities can be tackled by the development of novel intuitive human-robot interfaces. 

The objective of this research is indeed to explore and develop new human-robot interfaces that are intuitive and easy-to-use, minimizing the operator cognitive and physical workload while controlling a high-redundant robotic system. At the same time, the interfaces must deliver a certain level of robot autonomy, relieving the operator to take care of every minimal aspect of the task. 

Considering these key challenges, this research project explored new intuitive teleoperation paradigms.
One result is the development of the TelePhysicalOperation concept, an innovative teleoperation interface that combines the intuitiveness of a physical human-robot interaction maintaining the safety of controlling the robot from a distance.
The interface has been enriched with robot autonomy features to reduce the operator burden and to improve the performance of the task, in a shared-control fashion. For example, a mobile manipulator can generate arm and mobile base commands from a single input of the operator, relieving her/him to switch between arm and mobile base control. Thanks to another autonomy feature, the robot can bimanually keep the grasp on a transported load while the operator commands only the object velocities, without worrying about the grasping forces necessary to make the object to not fall. 

In general, it is not always straightforward for the operator to be aware about the status of the robot and of the task by visual information only, delivered by a monitor or by directly looking at the robot. For this reason, this research explored the importance of tactile feedback during the remote teleoperation. The TelePhysicalOperation interface has been improved by adding a feedback channel realized with the development of a customized sensorimotor interface. The integrated devices deliver indentation and vibrotactile feedback to the user's in response to inputs given to the robot and interactions with the robot's environment. This enhancement in the operator's experience has been validated involving novice TelePhysicalOperation users.

Another aspect of this work explored a teleoperation interface which relies on visual servoing guidance through a laser device. With the architecture developed, the operator, by pointing a laser emitter in the environment, is able to command target locations to even highly articulated robots effortlessly and efficiently for loco-manipulation tasks. The same interface has also been proved to be adaptable to a different scenario, an assistive one, where users with arm impairments can control a manipulator with the laser emitter worn on the head.    

In conclusion, this research has explored new kinds of interfaces to face the challenge of commanding complex robots but without overwhelming the operator with complicated user interfaces.